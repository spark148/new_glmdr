\documentclass[11pt]{extarticle}

\usepackage{indentfirst}
\usepackage{amsmath}
\usepackage[utf8]{inputenc}
\usepackage{natbib}
\usepackage{import}

\title{\textbf{TITLE}}
\author{Suyoung Park\thanks{Email: spark148@illinois.edu} \text{ }and Daniel Eck\\
\emph{University of Illinois at Urbana-Champaign}}
\date{Month 2020}

\begin{document}
\maketitle
\begin{abstract}
    Space for the Abstract.    
\end{abstract}
\smallskip
\noindent \textbf{Key Words:} list of keywords

\section{Introduction}
\subsection{Background and Literature}
\subsection{Motivations and Contributions}

\section{Problem Formulation}
\subsection{Preliminaries}
\subsubsection{Exponential Family}
An exponential family of distributions is a parametric statistical model having log likelihood
\begin{equation}\label{exp_ll}
    l(\theta)=y^T\theta-c(\theta)
\end{equation}
where $y$ is a vector statistics and $\theta$ is a vector parameter. This uses the convention that terms that do not contain the parameter can be dropped from a log likelihood; otherwise such terms might also appear in \ref{exp_ll}. A statistic $y$ and parameter $\theta$ that give a log likelihood of this form are called canonical or natural. The function $c$ is called the cumulant function of the family.

Let $\omega$ represent the full data, then the densities have the form
\begin{equation}\label{exp_pdf}
    f_{\theta}(\omega) = h(\omega)\exp(\langle\omega,\theta\rangle-c(\theta))
\end{equation}
and the word density here can refer to a probability mass function (PMF). The $h(\omega)$ arises from any term not containing the parameter that is dropped in going from log desntities to log likelihood.
We will write then density \ref{exp_pdf} as arising with respect to a generating measure $\lambda(d\omega)=h(\omega)d\omega$. Thus, the density for the exponential family is now
$$f_{\theta}(\omega) = \exp(\langle\omega,\theta\rangle-c(\theta))$$
where the cumulant function $c(\theta)$ is the log Laplace transformation of the measure $\lambda$
$$c(\theta)=\log(\int \exp(\omega,\theta)\lambda(d\omega).$$
The exponential family log likelihood takes the form 
\begin{equation}\label{exp_ll2}
    l(\theta)=\langle\omega,\theta\rangle-c(\theta)
\end{equation}




\subsubsection{Complete Separation}

\subsection{Problem Settings}

\section{Solution}
\subsection{Methodology}
\subsubsection{Wald's Confidence Interval}
\subsubsection{Likelihood based Confidence Interval}

\subsection{Analysis and Discussion}
\subsubsection{Results}
\subsubsection{Comparison with other methods}
\section{Conclusion}
\subsection{Background and Literature}

\bibliographystyle{plainnat}
\bibliography{Reference}

\section{Appendix}

\import{sections/}{./Figures/Endometrial_Table.tex}

\end{document}